%% Copernicus Publications Manuscript Preparation Template for LaTeX Submissions
%% ---------------------------------
%% This template should be used for copernicus.cls
%% The class file and some style files are bundled in the Copernicus Latex Package, which can be downloaded from the different journal webpages.
%% For further assistance please contact Copernicus Publications at: production@copernicus.org
%% https://publications.copernicus.org/for_authors/manuscript_preparation.html


%% Please use the following documentclass and journal abbreviations for discussion papers and final revised papers.

%% 2-column papers and discussion papers
\documentclass[gmd, manuscript]{copernicus}



%% Journal abbreviations (please use the same for discussion papers and final revised papers)


% Advances in Geosciences (adgeo)
% Advances in Radio Science (ars)
% Advances in Science and Research (asr)
% Advances in Statistical Climatology, Meteorology and Oceanography (ascmo)
% Annales Geophysicae (angeo)
% Archives Animal Breeding (aab)
% ASTRA Proceedings (ap)
% Atmospheric Chemistry and Physics (acp)
% Atmospheric Measurement Techniques (amt)
% Biogeosciences (bg)
% Climate of the Past (cp)
% DEUQUA Special Publications (deuquasp)
% Drinking Water Engineering and Science (dwes)
% Earth Surface Dynamics (esurf)
% Earth System Dynamics (esd)
% Earth System Science Data (essd)
% E&G Quaternary Science Journal (egqsj)
% European Journal of Mineralogy (ejm)
% Fossil Record (fr)
% Geochronology (gchron)
% Geographica Helvetica (gh)
% Geoscience Communication (gc)
% Geoscientific Instrumentation, Methods and Data Systems (gi)
% Geoscientific Model Development (gmd)
% History of Geo- and Space Sciences (hgss)
% Hydrology and Earth System Sciences (hess)
% Journal of Micropalaeontology (jm)
% Journal of Sensors and Sensor Systems (jsss)
% Magnetic Resonance (mr)
% Mechanical Sciences (ms)
% Natural Hazards and Earth System Sciences (nhess)
% Nonlinear Processes in Geophysics (npg)
% Ocean Science (os)
% Primate Biology (pb)
% Proceedings of the International Association of Hydrological Sciences (piahs)
% Scientific Drilling (sd)
% SOIL (soil)
% Solid Earth (se)
% The Cryosphere (tc)
% Weather and Climate Dynamics (wcd)
% Web Ecology (we)
% Wind Energy Science (wes)


%% \usepackage commands included in the copernicus.cls:
%\usepackage[german, english]{babel}
%\usepackage{tabularx}
%\usepackage{cancel}
%\usepackage{multirow}
%\usepackage{supertabular}
%\usepackage{algorithmic}
%\usepackage{algorithm}
%\usepackage{amsthm}
%\usepackage{float}
%\usepackage{subfig}
%\usepackage{rotating}
\usepackage{booktabs}

\renewcommand{\thefigure}{S\arabic{figure}}
\renewcommand{\thetable}{S\arabic{table}}
\renewcommand\thesection{S\arabic{section}}

\begin{document}

\title{Supplementary Information for \emph{FaIRv2.0.0: a generalised impulse-response model for climate uncertainty and future scenario exploration}}


% \Author[affil]{given_name}{surname}

\Author[1]{Nicholas J.}{Leach}
\Author[1]{Stuart}{Jenkins}
\Author[2,3]{Zebedee}{Nicholls}
\Author[4,5]{Christopher J.}{Smith}
\Author[1]{John}{Lynch}
\Author[1]{Michelle}{Cain}
\Author[1]{Tristram}{Walsh}
\Author[1]{Bill}{Wu}
\Author[6]{Junichi}{Tsutsui}
\Author[1,7]{Myles R.}{Allen}

\affil[1]{Department of Physics, Atmospheric, Oceanic, and Planetary Physics, University of Oxford, United Kingdom.}
\affil[2]{Australian--German Climate and Energy College, University of Melbourne, Australia.}
\affil[3]{School of Earth Sciences, University of Melbourne, Australia.}
\affil[4]{School of Earth and Environment, University of Leeds, Leeds, UK.}
\affil[5]{International Institute for Applied Systems Analysis, Laxenburg, Austria.}
\affil[6]{Environmental Science Laboratory, Central Research Institute of Electric Power Industry, Abiko-shi, Japan.}
\affil[7]{Environmental Change Institute, University of Oxford, Oxford, UK.}

%% The [] brackets identify the author with the corresponding affiliation. 1, 2, 3, etc. should be inserted.

%% If an author is deceased, please add a further affiliation and mark the respective author name(s) with a dagger, e.g. "\Author[2,$\dag$]{Anton}{Aman}" with the affiliations "\affil[2]{University of ...}" and "\affil[$\dag$]{deceased, 1 July 2019}"


\correspondence{Nicholas J. Leach (nicholas.leach@stx.ox.ac.uk)}

\runningtitle{TEXT}

\runningauthor{TEXT}





\received{}
\pubdiscuss{} %% only important for two-stage journals
\revised{}
\accepted{}
\published{}

%% These dates will be inserted by Copernicus Publications during the typesetting process.


\firstpage{1}

\maketitle

\copyrightstatement{@ Author(s) 2019. This work is distributed under
the Creative Commons Attribution 4.0 License.}

\begin{table}[t]
    \caption{Units used in FaIRv2.0.0-alpha when the default parameter set is used for each gas or aerosol species. Default forcing unit for all species is Wm$^{-2}$.}
    \label{tab:units}
    \input{tables/TabS1}
\end{table}
\clearpage
\section{FaIRv2.0.0 parameter defaults}
\begin{table}[h]
    \caption{FaIRv2.0.0 default parameter values. Available to download as a .csv at \url{TODO}.}
    \label{table:defaults}
    {\tiny
    \renewcommand{\arraystretch}{0.75}
    \input{tables/TabS2}
    }
\end{table}
\clearpage
%
\begin{table}[h]
    \caption{GWP metric for default parameter values computed against a baseline emission scenario that reproduces historical concentrations \citep{Meinshausen2017} up to 2014, and fixed at the 2014 level thereafter. These are calculated using the total change in ERF arising from a 1 t emission pulse of each emission type in 2015.}
    \label{table:GWPs}
    {\footnotesize
    \renewcommand{\arraystretch}{0.7}
    \input{tables/TabS3}
    }
\end{table}
\clearpage
%
\section{Radiative forcing of carbon dioxide, methane and nitrous oxide}
Here we compare the concentration-forcing relationships of CO$_2$, CH$_4$ and N$_2$O used in FaIRv2.0.0, which exclude the interaction terms between gases, to the standard simple formulae detailed in \cite{Etminan2016}. Figure \ref{fig:RF_etminan} shows a comparison of the Oslo line-by-line (OLBL) data from \citeauthor{Etminan2016} to both the \citeauthor{Etminan2016} formulae and those used in FaIRv2.0.0. The main difference between the relationships used in FaIRv2.0.0 and those in \citeauthor{Etminan2016} is the variance in the error when compared to the OLBL. The FaIRv2.0.0 relationships have a larger error variance at each CO$_2$ concentration than the \citeauthor{Etminan2016} formulae (Figure \ref{fig:RF_etminan}). This is due to the lack of interaction terms, and results in a maximum absolute error of 0.115 W m$^{-2}$ at concentrations of CO$_2=2000$ ppm, CH$_4=3500$ ppm, and N$_2$O$=525$ ppm (the green triangle in the far right-hand side subplot in Figure \ref{fig:RF_etminan}). We believe that in the context of other uncertainties associated with such a high concentration scenario this error is defensible. We note (as was done in \citeauthor{Etminan2016}) that the absolute uncertainty in the OLBL calculation is estimated to be 10\% for CO$_2$ and N$_2$O and 14\% for CH$_4$.\\
%
\begin{figure}[h]
    \includegraphics[width=\textwidth]{"figures/FigS1".pdf}
    \caption{Comparison of the CO$_2$, CH$_4$ and N$_2$O ERF relationships used in FaIRv2.0.0 to the simple formulae and OLBL data from \citep{Etminan2016}. We show absolute differences from OLBL approach, grouped by CO$_2$ concentration. Marker size indicates source (\citeauthor{Etminan2016} or FaIRv2.0.0-alpha), colour indicates CH$_4$ concentration, and style indicates N$_2$O concentration.}
    \label{fig:RF_etminan}
\end{figure}
\clearpage
\section{Methane lifetime model}
Here we illustrate the impact of including an interactive methane lifetime in FaIRv2.0.0. Running RCP8.5 \citep{Riahi2011} through FaIRv2.0.0-alpha, the evolution of methane lifetime over history and through 2100 is shown in Figure \ref{fig:CH4_lifetime}. The simulated methane lifetime in 2010 is 8.90 years, compared to 9.15 years in \cite{Holmes2013}; and the change in lifetime between 2010 and 2100 is 9.91 \% compared to 10.3 \%.
%
\begin{figure}[h]
    \includegraphics[width=\textwidth]{"figures/FigS2".pdf}
    \caption{Evolution of the methane lifetime in FaIRv2.0.0-alpha under historical concentrations \citep{Meinshausen2011c} to 2005 and the RCP8.5 scenario thereafter.}
    \label{fig:CH4_lifetime}
\end{figure}
\clearpage
%
\section{Non-linearities}
Non-linearities in FaIRv2.0.0 arise primarily from the atmospheric decay feedbacks within the carbon and methane cycles (ie. from the state-dependent adjustment factor, $\alpha(t)$). All other non-linearities are due to forcing agents with non-zero $f_1$ or $f_3$ parameters. Here we attempt to understand and characterise the non-linearities within FaIRv2.0.0 using a set of pulse-response experiments. We use pulse-response experiments as even in a simple model such as FaIRv2.0.0, the existing gas-cycle and forcing non-linearities can result in relatively complex behaviour. Using SSP2-45 as the reference scenario, we add in emission pulses at the present-day (2019) over several orders of magnitude (0.01 to 1000 GtCO$_2$-eq) to the input carbon dioxide and methane emission timeseries in turn. The results of these experiments are shown in Figure \ref{fig:nonlinearities}. In the next paragraph, we step through how to understand these results, but a key message to emphasize is that these non-linearities are not very significant on the whole, and especially on longer (centennial) timescales.\\
In terms of characterisation, we will step through each variable in turn, explaining the processes behind their behaviour. Throughout this section, we refer to the difference between the experiments and the reference as the \emph{anomaly}. In Figure \ref{fig:nonlinearities}, these are normalised by 1/(pulse size). We refer to the difference between the experimental anomalies relative to the smallest pulse experiment, normalised by 1/(pulse size) as the \emph{non-linearity anomaly}; if the model were linear, the non-linearity anomalies would be zero. For the CO$_2$ pulse experiments:
\begin{itemize}
\item $\alpha_{\text{CH}_4}$ (and hence the CH$_4$ lifetime) is reduced by a CO$_2$ pulse due to the overall increase in temperature. The non-linearity anomalies increase as the pulse size increases, mirorring the non-linearity anomalies in temperature.
\item  $\alpha_{\text{CO}_2}$ is increased by a CO$_2$ pulse due to the increased CO$_2$ uptake and temperature. The non-linearity anomalies increase as the pulse size increases.
\item  CH$_4$ concentrations are reduced by a CO$_2$ pulse due to the reduction in CH$_4$ lifetime. The non-linearity anomalies increase (ie. the overall anomaly becomes less negative) as the pulse size increases.
\item  CO$_2$ concentrations are increased by a CO$_2$ pulse due to the increase in $\alpha_{\text{CO}_2}$. The non-linearity anomalies increase as the pulse size increases, hence CO$_2$ concentrations are super-linear with emissions. However, the magnitude of the differences due to non-linearities are still small: $<0.5$ ppm for pulses lower than 100 GtCO$_2$.
\item  CH$_4$ forcing anomalies and non-linearity anomalies follow the same behaviour as CH$_4$ concentrations.
\item  CO$_2$ forcing anomalies follow the same behaviour as CH$_4$ concentration anomalies. However, the non-linearity anomalies display different behaviour. Since CO$_2$ forcing is sub-linear with concentrations, the non-linearity anomaly is lower the larger the pulse size (the concentration-forcing sub-linearity dominates the emission-concentration super-linearity).
\item  Temperature anomalies are, as expected, positive for a CO$_2$ pulse. However, temperature non-linearity anomalies follow the same behaviour as CO$_2$ forcing, and are therefore sub-linear with CO$_2$ emissions. We note that the magnitude of the differences arising due to model non-linearity are $<0.005$ K on all timescales, and $<0.001$ K on centennial timescales for pulse sizes of $<100$ GtCO$_2$.
\end{itemize}
The CH$_2$ pulse experiments can be analysed in an identical fashion. An additional feature of the behaviour during these experiments is that the sub-linearity of CH$_4$ forcing with concentrations dominates on very short timescales (meaning that the initial temperature non-linearity anomaly is negative), but for timescales of $> 10$ years, the super-linearity of CH$_4$ concentrations with emissions dominates; resulting in an overall long-term super-linear temperature response with CH$_4$ emissions. Note that even for emission pulses of CH$_4$, centennial non-linearities are driven by the carbon-cycle due to the short atmospheric lifetime of CH$_4$. However, we again stress that the overall differences arising due to model non-linearity are $<0.05$ K on all timescales, and $<0.001$ K on centennial timescales for pulse sizes of $<100$ GtCO$_2$-eq.\\
Overall, the non-linearities appear insignificant for pulses of less than 10 GtCO$_2$-eq. It is important to note that for realstic emission scenarios these non-linearities would grow over time, so over long time periods non-linearities may be important even if the emission rates are significantly lower than 10 GtCO$_2$-eq.
%
\begin{figure}[h]
    \includegraphics[width=\textwidth]{"figures/FigS3".pdf}
    \caption{Demonstration of non-linear behaviour in FaIRv2.0.0 using pulse emission experiments. Top two rows show results from CO$_2$ pulse experiments, and bottom two show results from CH$_4$ experiments. The columns show, in order: CH$_4$ lifetime adjustment factor, CO$_2$ lifetime adjustment factor, CH$_4$ concentrations, CO$_2$ concentrations, CH$_4$ forcing, CO$_2$ forcing, and temperature response. Rows 1 and 3 show anomalies with respect to the reference (no pulse) experiment, scaled by 1/(pulse size). Rows 2 and 4 (marked as "relative") show the scaled anomalies relative to the anomaly for a pulse of size 0.01 GtCO$_2$-eq. The ratio of rows 1 and 2 (or 3 and 4) gives the fractional contribution of the nonlinearity relative to the size of the anomaly for each experiment. Non-linearities are not visible at these scales for pulses of less than 1 GtCO$_2$-eq.}
    \label{fig:nonlinearities}
\end{figure}
%
\section{CMIP6 data pre-processing}
CMIP6 data throughout this study is pre-processed as described in \citet{Nicholls2021}, using a 21-year-running mean in the piControl experiment to normalise the data and calculate anomalies. The number of ensemble members per model for each experiment is given below in Table \ref{table:mem_numbers}.
\begin{table}[h]
    \caption{Number of ensemble members per CMIP6 model in the idealised experiments used to tune the FaIRv2.0.0-alpha climate response in Section 3.1; and in the SSPs displayed in Figure 10.}
    \label{table:mem_numbers}
    {\scriptsize
    \renewcommand{\arraystretch}{0.75}
    \input{tables/TabS4}
    }
\end{table}
\clearpage
%
\section{Energy balance model parameters}
Using the method described in \cite{Cummins2020}, we tune parameters to CMIP6 models for a 3-box version of the energy balance model described in \cite{Geoffroy2013}, with the ocean heat uptake efficacy factor as \cite{Winton2010} included. 
\begin{table}[h]
    \caption{CMIP6 tuned 3-box energy balance model. Nomenclature as \citet{Cummins2020}.}
    \label{table:EB_p}
    \renewcommand{\arraystretch}{0.9}
    \input{tables/TabS5}
\end{table}
\clearpage
%
\section{Global Warming Index calculation}
The Global Warming Index follows the methodology in \cite{Haustein2016}, but updates several components. We carry out the calculation using 6 observational warming products \citep{Lenssen2019,Cowtan2014,Vose2012,Morice2011,Rohde2013,Morice2020}, incorporating observational uncertainty either through the provided ensemble product, or if none exists, through the HadCRUT5 ensemble errors. We then generate 5000 realisations of historical ERF as follows. For all forcings excluding aerosol forcing, we take the best-estimate historical timeseries from \cite{Smith2020c}, and scale them by factors drawn from the distributions detailed in table 6. For aerosol forcing, we take the best-estimate historical timeseries of ERFaci and ERFari, and scale them by scaling factors drawn from a skew-normal distribution that matches the quantiles of the constrained ERFaci and ERFari distributions stated in table 4 of \cite{Smith2020a}. We consider 18 different response model parameterisations, spanning the ranges of possible realised warming fraction \citep{Millar2015} and response timescale \citep{Geoffroy2013}. Finally, we include uncertainty due to internal variability through timeseries from the piControl experiment of different CMIP6 models, rejecting models with a too large drift (here ``too large'' is $>\pm0.15$ K / century); resulting in 102 different representations of internal variability (two random samples per model). Combining these sources of uncertainty gives a 1,836,000,000 member ensemble of the global warming index. For each observational product, we then randomly subsample 500,000,000 members which are used to estimate the distribution of the current level and rate of anthropogenic warming, and thus the FULL ensemble selection probabilities.
\clearpage
%
\section{Alternate baseline warming projections}
\begin{table}[h]
    \caption{Projections of future warming as Table 10, but relative to a pre-industrial baseline period of 1850-1900 for comparison with \citet{Tokarska2020a,Ribes2021}. Shown are end of century warming (2081-2100), 2100 warming, and peak warming.}
    \label{table:alt_proj}
    \input{tables/TabS6}
\end{table}
\clearpage
%
\section{Short note on Implementation of FaIRv2.0.0}
In this section, we outline the details of how we have implemented the development version of the FaIRv2.0.0 model (FaIRv2.0.0-alpha) in python. We note that these notes may not be relevant for all programming languages, but hope that they are useful regardless, especially where a forward difference timestepping scheme is used.\\
In our implementation, we use a forward difference scheme, in which output variables are assumed to be intra-timestep averages, and auxiliary variables (such as $R$ or $G_a$) are calculated instantaneously at the start of each timestep. Within each timestep, we carry out the following sequence of computations:\\
\begin{enumerate}
\item Calculate $\alpha(t)$ using equation 3 with variables values from the previous timestep.
\item Calculate $C(\overline{t})$ by
\begin{enumerate}
\item calculating $R_i(t) = E(t)*a_i*\alpha(t)*\tau_i*[1-exp(dt/(\alpha(t)*\tau_i))]/dt + R(t-1) * exp(dt/(\alpha(t)*\tau_i))$
\item $G_a = \sum_{i=1}^n{R_i(t)}$
\item $C(\overline{t}) = C_0 + (G_a(t)+G_a(t-1))/2$
\end{enumerate}
\item Calculate $F(\overline{t})$ using equation 4.
\item Calculate $T(\overline{t})$ by
\begin{enumerate}
\item $S_j(t) = q_j * F(\overline{t}) * [1-exp(-dt/d_i)] + S_j(t-1) * exp(-dt/d_i)$
\item $T(\overline{t}) = \sum_{j=1}^3{S_j(t-1)+S_j(t)/2}$
\end{enumerate}
\item All output variables have now been computed, and the process restarts.
\end{enumerate}
For clarity, we also provide a schematic, Figure \ref{fig:implementation}, indicating at which point in each timestep each variable is assumed to reside in our implementation. In addition to our timestepping scheme, the other key feature of our implementation (and important property of FaIRv2.0.0) is that it is fully vectorised. Variables are represented by - and computations run over - multi-dimensional arrays. In our implementation, these arrays have dimensions for (where required): emission/ forcing scenario, gas/forcing parameter set, climate response parameter set, gas species, time, and any additional required dimensions (reservoirs for the carbon cycle or thermal boxes for the climate response). For example, the size of the array containing modelled concentrations in an emission driven run would be: $\text{S}\times\text{P}\times\text{C}\times\text{G}\times\text{t}$, where S is the number of scenarios, P the number of gas parameter sets, C the number of climate response sets, G the number of gas species and t the total number of timesteps. This ability of FaIRv2.0.0 to be vectorised significantly reduces computation time for array-focused languages such as Python or MatLab. However, an important point to note is that this “complete” vectorisation results in the stored variable arrays becoming large very rapidly (eg. a tenfold increase in each of the S, P and C dimensions would result in a thousandfold increase in the size of the stored arrays, and corresponding memory required). It is therefore important to check that the total size of the stored arrays won’t exceed the total size of the available memory on the machine. It is possible that we could implement some form of “lazy” computation to alleviate this issue, for example by using Dask\footnote{https://dask.org/}, but we have not done so in the FaIRv2.0.0-alpha implementation accompanying this paper.
%
\begin{figure}[h]
    \includegraphics[width=\textwidth]{"figures/FigS4".pdf}
    \caption{Variable residence within FaIRv2.0.0-alpha timestepping scheme.}
    \label{fig:implementation}
\end{figure}
\clearpage
%

%% The following commands are for the statements about the availability of data sets and/or software code corresponding to the manuscript.
%% It is strongly recommended to make use of these sections in case data sets and/or software code have been part of your research the article is based on.

% \codeavailability{TEXT} %% use this section when having only software code available


% \dataavailability{TEXT} %% use this section when having only data sets available


\codedataavailability{FaIRv2.0.0-alpha code and the code used to produce the figures is publicly available at \url{TODO}. All data used in this study is publicly available at the relevant cited sources.} %% use this section when having data sets and software code available


% \sampleavailability{TEXT} %% use this section when having geoscientific samples available


% \videosupplement{TEXT} %% use this section when having video supplements available


% \appendix
% \section{}    %% Appendix A

% \subsection{}     %% Appendix A1, A2, etc.


\noappendix       %% use this to mark the end of the appendix section

%% Regarding figures and tables in appendices, the following two options are possible depending on your general handling of figures and tables in the manuscript environment:

%% Option 1: If you sorted all figures and tables into the sections of the text, please also sort the appendix figures and appendix tables into the respective appendix sections.
%% They will be correctly named automatically.

%% Option 2: If you put all figures after the reference list, please insert appendix tables and figures after the normal tables and figures.
%% To rename them correctly to A1, A2, etc., please add the following commands in front of them:

\appendixfigures  %% needs to be added in front of appendix figures

\appendixtables   %% needs to be added in front of appendix tables

%% Please add \clearpage between each table and/or figure. Further guidelines on figures and tables can be found below.



% \authorcontribution{NJL, SJ and MRA conceived the study. NJL and SJ wrote the model code, and BW helped tune model parameters. JT provided CMIP6 response parameters. JL and MC advised on model uses and tested the model. NJL, CJS, ZN, JL and MRA wrote the manuscript.} %% this section is mandatory

% \competinginterests{We declare that we have no competing interests.} %% this section is mandatory even if you declare that no competing interests are present

% \disclaimer{TEXT} %% optional section

% \begin{acknowledgements}
% We acknowledge the World Climate Research Programme, which, through its Working Group on Coupled Modelling, coordinated and promoted both CMIP5 and CMIP6.
% \end{acknowledgements}




%% REFERENCES

%% The reference list is compiled as follows:

% \begin{thebibliography}{}

% \bibitem[AUTHOR(YEAR)]{LABEL1}
% REFERENCE 1

% \bibitem[AUTHOR(YEAR)]{LABEL2}
% REFERENCE 2

% \end{thebibliography}

%% Since the Copernicus LaTeX package includes the BibTeX style file copernicus.bst,
%% authors experienced with BibTeX only have to include the following two lines:
%%
\bibliographystyle{copernicus}
\bibliography{reference_list.bib}
%%
%% URLs and DOIs can be entered in your BibTeX file as:
%%
%% URL = {http://www.xyz.org/~jones/idx_g.htm}
%% DOI = {10.5194/xyz}


%% LITERATURE CITATIONS
%%
%% command                        & example result
%% \citet{jones90}|               & Jones et al. (1990)
%% \citep{jones90}|               & (Jones et al., 1990)
%% \citep{jones90,jones93}|       & (Jones et al., 1990, 1993)
%% \citep[p.~32]{jones90}|        & (Jones et al., 1990, p.~32)
%% \citep[e.g.,][]{jones90}|      & (e.g., Jones et al., 1990)
%% \citep[e.g.,][p.~32]{jones90}| & (e.g., Jones et al., 1990, p.~32)
%% \citeauthor{jones90}|          & Jones et al.
%% \citeyear{jones90}|            & 1990



%% FIGURES

%% When figures and tables are placed at the end of the MS (article in one-column style), please add \clearpage
%% between bibliography and first table and/or figure as well as between each table and/or figure.


%% ONE-COLUMN FIGURES

%%f
%\begin{figure}[t]
%\includegraphics[width=8.3cm]{FILE NAME}
%\caption{TEXT}
%\end{figure}
%
%%% TWO-COLUMN FIGURES
%
%%f
%\begin{figure*}[t]
%\includegraphics[width=12cm]{FILE NAME}
%\caption{TEXT}
%\end{figure*}
%
%
%%% TABLES
%%%
%%% The different columns must be seperated with a & command and should
%%% end with \\ to identify the column brake.
%
%%% ONE-COLUMN TABLE
%
%%t
%\begin{table}[t]
%\caption{TEXT}
%\begin{tabular}{column = lcr}
%\tophline
%
%\middlehline
%
%\bottomhline
%\end{tabular}
%\belowtable{} % Table Footnotes
%\end{table}
%
%%% TWO-COLUMN TABLE
%
%%t
%\begin{table*}[t]
%\caption{TEXT}
%\begin{tabular}{column = lcr}
%\tophline
%
%\middlehline
%
%\bottomhline
%\end{tabular}
%\belowtable{} % Table Footnotes
%\end{table*}
%
%%% LANDSCAPE TABLE
%
%%t
%\begin{sidewaystable*}[t]
%\caption{TEXT}
%\begin{tabular}{column = lcr}
%\tophline
%
%\middlehline
%
%\bottomhline
%\end{tabular}
%\belowtable{} % Table Footnotes
%\end{sidewaystable*}
%
%
%%% MATHEMATICAL EXPRESSIONS
%
%%% All papers typeset by Copernicus Publications follow the math typesetting regulations
%%% given by the IUPAC Green Book (IUPAC: Quantities, Units and Symbols in Physical Chemistry,
%%% 2nd Edn., Blackwell Science, available at: http://old.iupac.org/publications/books/gbook/green_book_2ed.pdf, 1993).
%%%
%%% Physical quantities/variables are typeset in italic font (t for time, T for Temperature)
%%% Indices which are not defined are typeset in italic font (x, y, z, a, b, c)
%%% Items/objects which are defined are typeset in roman font (Car A, Car B)
%%% Descriptions/specifications which are defined by itself are typeset in roman font (abs, rel, ref, tot, net, ice)
%%% Abbreviations from 2 letters are typeset in roman font (RH, LAI)
%%% Vectors are identified in bold italic font using \vec{x}
%%% Matrices are identified in bold roman font
%%% Multiplication signs are typeset using the LaTeX commands \times (for vector products, grids, and exponential notations) or \cdot
%%% The character * should not be applied as mutliplication sign
%
%
%%% EQUATIONS
%
%%% Single-row equation
%
%\begin{equation}
%
%\end{equation}
%
%%% Multiline equation
%
%\begin{align}
%& 3 + 5 = 8\\
%& 3 + 5 = 8\\
%& 3 + 5 = 8
%\end{align}
%
%
%%% MATRICES
%
%\begin{matrix}
%x & y & z\\
%x & y & z\\
%x & y & z\\
%\end{matrix}
%
%
%%% ALGORITHM
%
%\begin{algorithm}
%\caption{...}
%\label{a1}
%\begin{algorithmic}
%...
%\end{algorithmic}
%\end{algorithm}
%
%
%%% CHEMICAL FORMULAS AND REACTIONS
%
%%% For formulas embedded in the text, please use \chem{}
%
%%% The reaction environment creates labels including the letter R, i.e. (R1), (R2), etc.
%
%\begin{reaction}
%%% \rightarrow should be used for normal (one-way) chemical reactions
%%% \rightleftharpoons should be used for equilibria
%%% \leftrightarrow should be used for resonance structures
%\end{reaction}
%
%
%%% PHYSICAL UNITS
%%%
%%% Please use \unit{} and apply the exponential notation


\end{document}

